
\documentclass[jmkletter]{scrlttr2}
\usepackage[english]{babel}
\usepackage{blindtext}

\setkomavar{subject}{}

\begin{document}

  
  \begin{letter}{Editorial Board\\Nature Geoscience
}
    \opening{Dear Editors,}

Please find uploaded to your system our manuscript ``Submesoscale streamers exchange water on the north wall of the Gulf Stream''.  

The Gulf Stream is one of the most studied currents in the world, yet oceanographers still struggle to understand how it entrains water and exchanges properties with the ocean gyres it delineates. Lateral mixing processes are inferred from large-scale budgets, but have never been directly observed.  Here we present direct observations, a dye release, and modelling results of one such process:  partially mixed water at the ``North Wall'' of the Gulf Stream peals off in 50-km long, 5-km wide ``streamers'', and cool fresh water is entrained in its place.  Remarkably, this process serves to preferentially remove water from the Gulf Stream front that has undergone small scale mixing, and by drawing in fresh water to replace it serves to resharpen the front.  

The Gulf Stream has been well-studied, and hints of these streamers have been observed in satellites before, and their presence inferred from subsurface floats that were observed to peel off the north side of the Gulf Stream with some regularity.  The field campaign reported here was unique in our finescale sampling of the front using two co-ordinated vessels that tracked a Lagrangian float as it moved downstream.  This allowed us  to make finescale ``tomographic slices'' of the underwater structure that are unprecedented.  They give the new conclusions that:
\begin{itemize}
  \item The Gulf Stream front stays remarkably sharp for hundreds of kilometers
  \item Mixing at the front (which is believed to be vigorous) produces partially mixed water, a substantial portion of which is advected away in the streamers
  \item Velocity measurements (and model results) indicate the rate of detrainment of this water.
  \item Dye and model results indicate that fresh water is entrained (i.e. the exchange is two-way), helping to resharpen the front.
  \item A rough budget indicates that this process  entrains enough fresh water to match large-scale budgets that show the Gulf Stream is freshening as it moves donwstream.  
\end{itemize}

In short, this work ties together fragmentary hints about the streamer process from the last 50 years, showing definitive depth-space maps of the process that link the underwater structure to the surface.  It also provides the first quantitative estimates of the strength of exchange mediated by the streamers.  The phenomenology will be of wide interest to earth scientists, and the details of importance to oceanographers who seek to better understand the mechanisms of mixing at strong oceanic fronts.  


    \closing{Sincerely Yours,}
  \end{letter}
 
\end{document}
 
